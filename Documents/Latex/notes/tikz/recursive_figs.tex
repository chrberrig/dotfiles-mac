\documentclass[a4paper,twoside]{article}
\usepackage[utf8]{inputenc}
%\usepackage{natbib}            % for bibtex
\usepackage[style=phys]{biblatex}                       % for biblatex
\bibliography{/Users/chrberrig/Documents/LaTeX/bib.bib} % for biblatex
\usepackage{amsmath}
\usepackage{amsfonts}
\usepackage{amssymb}
\usepackage{physics}
%\usepackage[danish]{babel} 
\usepackage{graphicx}
%\usepackage{subcaption}
%\usepackage[rmargin=2.7cm,lmargin=2.7cm,bmargin=2.5cm,tmargin=2.5cm]{geometry}
\usepackage{float}
\usepackage{tikz}
\usetikzlibrary{lindenmayersystems}
%\usepackage{minted}
%\usepackage{listings}
%\usepackage{tkz-euclide} %for euclidian constructions in Tikz. 
%\usepackage{pgfplots}
\setlength{\parindent}{0pt}
\setlength{\parskip}{1ex plus 0.5ex minus 0.2ex}

\newcommand{\mtx}[1]{\underline{\underline{\textbf{#1}}}}
\newcommand{\diff}[2]{\dfrac{d #1}{d #2}}
\newcommand{\pdiff}[2]{\dfrac{\partial #1}{\partial #2}}

\title{Recursive figures w. tikz (using lindenmeyer systems)}
\author{Christian Berrig}
\date{\today}

\begin{document}

\maketitle

\section{Sierpinski Triangle}
\def\trianglewidth{2cm}%
\pgfdeclarelindenmayersystem{Sierpinski triangle}{
    \symbol{X}{\pgflsystemdrawforward}
    \symbol{Y}{\pgflsystemdrawforward}
    \rule{X -> X-Y+X+Y-X}
    \rule{Y -> YY}
}%
\foreach \level in {0,...,5}{%
\tikzset{
    l-system={step=\trianglewidth/(2^\level), order=\level, angle=-120}
}%
\begin{tikzpicture}
    \fill [black] (0,0) -- ++(0:\trianglewidth) -- ++(120:\trianglewidth) -- cycle;
    \draw [draw=none] (0,0) l-system
    [l-system={Sierpinski triangle, axiom=X},fill=white];
\end{tikzpicture}
}%

\begin{verbatim}
\def\trianglewidth{2cm}%
\pgfdeclarelindenmayersystem{Sierpinski triangle}{
    \symbol{X}{\pgflsystemdrawforward}
    \symbol{Y}{\pgflsystemdrawforward}
    \rule{X -> X-Y+X+Y-X}
    \rule{Y -> YY}
}%
\foreach \level in {0,...,5}{%
\tikzset{
    l-system={step=\trianglewidth/(2^\level), order=\level, angle=-120}
}%
\begin{tikzpicture}
    \fill [black] (0,0) -- ++(0:\trianglewidth) -- ++(120:\trianglewidth) -- cycle;
    \draw [draw=none] (0,0) l-system
    [l-system={Sierpinski triangle, axiom=X},fill=white];
\end{tikzpicture}
}%
\end{verbatim}

%\begin{minted}[language=tex]
%\def\trianglewidth{2cm}%
%\pgfdeclarelindenmayersystem{Sierpinski triangle}{
%    \symbol{X}{\pgflsystemdrawforward}
%    \symbol{Y}{\pgflsystemdrawforward}
%    \rule{X -> X-Y+X+Y-X}
%    \rule{Y -> YY}
%}%
%\foreach \level in {0,...,5}{%
%\tikzset{
%    l-system={step=\trianglewidth/(2^\level), order=\level, angle=-120}
%}%
%\begin{tikzpicture}
%    \fill [black] (0,0) -- ++(0:\trianglewidth) -- ++(120:\trianglewidth) -- cycle;
%    \draw [draw=none] (0,0) l-system
%    [l-system={Sierpinski triangle, axiom=X},fill=white];
%\end{tikzpicture}
%}%
%\end{minted}

\section{Sierpinski Triangle2}
\def\trianglewidth{2cm}%
\pgfdeclarelindenmayersystem{Sierpinski triangle2}{
    \symbol{X}{\pgflsystemdrawforward}
    \symbol{Y}{\pgflsystemdrawforward}
    \rule{X -> Y-X-Y}
    \rule{Y -> X+Y+X}
}%
\foreach \level in {0,...,5}{%
\begin{tikzpicture}
    \draw [black]
    [l-system={Sierpinski triangle2, step=\trianglewidth/(2^\level), order=\level, angle=60, axiom=X}]
    lindenmayer system;
\end{tikzpicture}
}%

\begin{verbatim}
\def\trianglewidth{2cm}%
\pgfdeclarelindenmayersystem{Sierpinski triangle}{
    \symbol{X}{\pgflsystemdrawforward}
    \symbol{Y}{\pgflsystemdrawforward}
    \rule{X -> X-Y+X+Y-X}
    \rule{Y -> YY}
}%
\foreach \level in {0,...,5}{%
\tikzset{
    l-system={step=\trianglewidth/(2^\level), order=\level, angle=-120}
}%
\begin{tikzpicture}
    \fill [black] (0,0) -- ++(0:\trianglewidth) -- ++(120:\trianglewidth) -- cycle;
    \draw [draw=none] (0,0) l-system
    [l-system={Sierpinski triangle, axiom=X},fill=white];
\end{tikzpicture}
}%
\end{verbatim}


\section{Dragon curve}
\pgfdeclarelindenmayersystem{Dragon curve}{
\symbol{X}{\pgflsystemdrawforward}
\symbol{+}{\pgflsystemturnright}% Explicitly define + and - symbols.
\symbol{-}{\pgflsystemturnleft}
\rule{A-> A+BX+}
\rule{B-> -XA−B}
}
\foreach \level in {1,...,2}{
\begin{tikzpicture}
\draw[l-system={Dragon curve, step=2cm/2^\level, axiom=XA, order=\level, angle=90}]
l-system;
\end{tikzpicture}
}

\pgfdeclarelindenmayersystem{Hilbert curve}{
	\symbol{X}{\pgflsystemdrawforward}
	\symbol{+}{\pgflsystemturnright}% Explicitly define + and - symbols.
	\symbol{-}{\pgflsystemturnleft}
	\rule{A-> +BX-AXA-XB+}
	\rule{B-> -AX+BXB+XA-}
}
\foreach \level in {1,...,7}{%
\begin{tikzpicture}
\draw[l-system={Hilbert curve, step=4cm/2^\level, axiom=A, order=\level, angle=90}]
l-system;
\end{tikzpicture}
}

\section{Sierpinski Hexagon}
\def\hexagwidth{2cm}%
\pgfdeclarelindenmayersystem{Sierpinski hexagon}{
  \symbol{X}{\pgflsystemdrawforward}
    \symbol{Y}{\pgflsystemmoveforward\pgflsystemmoveforward\pgflsystemmoveforward}
    \rule{X -> X+X+X+X+X+X+Y}
    \rule{Y -> YYY}
}%
\foreach \level in {1,...,4}{%
\tikzset{
    l-system={step=\hexagwidth/3^\level, order=\level, angle=60}
}%
\begin{tikzpicture}
  \fill (0,0) l-system [l-system={Sierpinski hexagon, axiom=X}] ;
\end{tikzpicture}
}%

\begin{verbatim}
\def\hexagwidth{2cm}%
\pgfdeclarelindenmayersystem{Sierpinski hexagon}{
  \symbol{X}{\pgflsystemdrawforward}
    \symbol{Y}{\pgflsystemmoveforward\pgflsystemmoveforward\pgflsystemmoveforward}
    \rule{X -> X+X+X+X+X+X+Y}
    \rule{Y -> YYY}
}%
\foreach \level in {1,...,4}{%
\tikzset{
    l-system={step=\hexagwidth/3^\level, order=\level, angle=60}
}%
\begin{tikzpicture}
  \fill (0,0) l-system [l-system={Sierpinski hexagon, axiom=X}] ;
\end{tikzpicture}
}%
\end{verbatim}


\section{Iterated Triangle}
\def\trianglewidth{2cm}%
\pgfdeclarelindenmayersystem{Iterated triangle}{
    \symbol{X}{\pgflsystemdrawforward}
    \symbol{Y}{\pgflsystemdrawforward}
    \rule{X -> Y-Y+X+Y-Y}
    \rule{Y -> YY}
}%
\foreach \level in {0,...,5}{%
\tikzset{
    l-system={step=\trianglewidth/(2^\level), order=\level, angle=-120}
}%
\begin{tikzpicture}
    \fill [black] (0,0) -- ++(0:\trianglewidth) -- ++(120:\trianglewidth) -- cycle;
    \draw [draw=none] (0,0) l-system
    [l-system={Iterated triangle, axiom=X},fill=white];
\end{tikzpicture}
}%

\begin{verbatim}
\def\trianglewidth{2cm}%
\pgfdeclarelindenmayersystem{Iterated triangle}{
    \symbol{X}{\pgflsystemdrawforward}
    \symbol{Y}{\pgflsystemdrawforward}
    \rule{X -> Y-Y+X+Y-Y}
    \rule{Y -> YY}
}%
\foreach \level in {0,...,5}{%
\tikzset{
    l-system={step=\trianglewidth/(2^\level), order=\level, angle=-120}
}%
\begin{tikzpicture}
    \fill [black] (0,0) -- ++(0:\trianglewidth) -- ++(120:\trianglewidth) -- cycle;
    \draw [draw=none] (0,0) l-system
    [l-system={Iterated triangle, axiom=X},fill=white];
\end{tikzpicture}
}%
\end{verbatim}

% \bibliographystyle{plain}                                     % for bibtex
% \bibliography{/Users/chrberrig/Documents/LaTeX/bib.bib}       % for bibtex
\printbibliography      % for biblatex

\end{document}
