\documentclass[a4paper,twoside]{beamer}
\usepackage[utf8]{inputenc}
%\usepackage{natbib}            % for bibtex
\usepackage[style=phys]{biblatex}                       % for biblatex
\bibliography{/Users/chrberrig/Documents/LaTeX/bib.bib} % for biblatex
\usepackage[affil-it]{authblk}    % for author affiliations
\usepackage{amsmath}
\usepackage{amsfonts}
\usepackage{amssymb}
\usepackage{amsthm}
\usepackage{physics}
%\usepackage[danish]{babel} 
\usepackage{graphicx}
\usepackage{subcaption}
\usepackage[rmargin=2.7cm,lmargin=2.7cm,bmargin=2.5cm,tmargin=2.5cm]{geometry}
\usepackage{float}
\usepackage{tikz}
\usepackage{tkz-euclide} %for euclidian constructions in Tikz. 
\usepackage{pgfplots}
\setlength{\parindent}{0pt}
\setlength{\parskip}{1ex plus 0.5ex minus 0.2ex}

\newcommand{\mtx}[1]{\underline{\underline{\textbf{#1}}}}
\newcommand{\diff}[2]{\dfrac{d #1}{d #2}}
\newcommand{\pdiff}[2]{\dfrac{\partial #1}{\partial #2}}

\title{}
\author[1]{Christian Berrig%
\thanks{Electronic address: \texttt{chrber@fysik.dtu.dk}}}
\affil[1]{BigQ, QPIT, DTU}
\date{\today}

\begin{document}

\maketitle


\begin{frame}
\frametitle{There Is No Largest Prime Number}
\framesubtitle{The proof uses \textit{reductio ad absurdum}.}
\begin{theorem}
There is no largest prime number.
\end{theorem}
\begin{proof}
\begin{enumerate}
\item<1-| alert@1> Suppose $p$ were the largest prime number.
\item<2-> Let $q$ be the product of the first $p$ numbers.
\item<3-> Then $q+1$ is not divisible by any of them.
\item<1-> But $q + 1$ is greater than $1$, thus divisible by some primenumber not in the first $p$ numbers.\qedhere
\end{enumerate}
\end{proof}
\end{frame}

% \bibliographystyle{plain}                                     % for bibtex
% \bibliography{/Users/chrberrig/Documents/LaTeX/bib.bib}       % for bibtex
\printbibliography      % for biblatex

\end{document}
